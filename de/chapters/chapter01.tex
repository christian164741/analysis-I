\chapter{Mengen \& Abbildungen}



\subsection{Einführung in den Kurs – Analysis}


In diesem Kurs werden die Grundlagen der Analysis~I behandelt, also die Untersuchung von Funktionen einer Variablen.
Im Mittelpunkt stehen die Begriffe Grenzwerte, Stetigkeit, Differenzierbarkeit, Reihen (insbesondere Potenz- und Taylorreihen),
Fourieranalyse, vollständige Induktion sowie Integration.

\medskip
\noindent\textbf{Aufbau des Kurses:} Der Kurs ist in 15 Kapitel gegliedert.

% chapters/02_mengen.tex
\subsection{Mengen}

Georg Cantor beschrieb 1895 eine Menge wie folgt:
\begin{quote}
	Eine Menge ist eine wohldefinierte Zusammenfassung verschiedener Objekte zu einem Ganzen.
\end{quote}

\subsection*{Beispiele}

\begin{itemize}
	\item $S:=\{\text{Berlin},\ \text{Hamburg},\ \text{Köln},\ \text{Hannover}\}$
	\item $\mathbb{R}:=\{x \mid x \text{ ist eine reelle Zahl}\}$ \quad (umgangssprachlich: \glqq alle Kommazahlen\grqq)
	\item $\sqrt{2}\in\mathbb{R}$, \quad $\{\sqrt{2}\}\subset \mathbb{R}$
	\item $\emptyset := \{\}\subset \mathbb{R}$
	\item $\infty \notin \mathbb{R}$
	\item $\{1,2,7,\mathrm{Berlin}\}\notin \mathbb{R}$
\end{itemize}

\subsection*{Wichtige Zahlenmengen}

\begin{align*}
	\mathbb{N} &:= \{0,1,2,3,\ldots\} \\
	\mathbb{Z} &:= \{\ldots,-2,-1,0,1,2,\ldots\} \\
	\mathbb{Q} &:= \left\{\frac{p}{q}\ \middle|\ p\in\mathbb{Z},\ q\in\mathbb{N}\setminus\{0\}\right\} \\
	\mathbb{R} &:= \text{Menge der reellen Zahlen} \\
	\mathbb{C} &:= \{a+\mathrm{i}b \mid a,b\in\mathbb{R},\ \mathrm{i}^2=-1\}
\end{align*}

\noindent\textbf{Achtung:} Oft werden die natürlichen Zahlen auch so definiert:
\begin{align*}
	\mathbb{N} &:= \{1,2,3,\ldots\}, \\
	\mathbb{N}_0 &:= \{0,1,2,\ldots\}.
\end{align*}

\subsection*{Eigenschaft}
\[
\mathbb{N}\subset\mathbb{Z}\subset\mathbb{Q}\subset\mathbb{R}\subset\mathbb{C}.
\]

\subsection*{Mathematische Schreibweisen}
\begin{itemize}
	\item $\in$ \quad ist Element von
	\item $\notin$ \quad ist kein Element von
	\item $\emptyset$ \quad leere Menge
	\item $|$ \quad \glqq so dass\grqq{} (z.\,B. in Mengen-Schreibweisen)
\end{itemize}

\subsection{Intervalle}
Seien $a,b\in\mathbb{R}$.

\subsubsection*{Abgeschlossenes Intervall}
\[
[a,b] := \{x\in\mathbb{R}\mid a\le x \le b\}.
\]

\begin{center}
	\begin{tikzpicture}
		\node at (2,1) [rotate=0,color=red] {$[$};
		\node at (2,0.5) [rotate=0, color=red] {$a$};
		\draw[->] (0,1) -- (6,1) node [below]{$\mathbb{R}$};
		\draw  (2,1) -- (4,1)[ color=red] ;
		\node at (4,1) [rotate=0,color=red] {$]$};
		\node at (4,0.5) [rotate=0, color=red] {$b$};
	\end{tikzpicture}
	
	z.\,B.: $[1,3]$
	
	\begin{tikzpicture}
		\node at (2,1) [rotate=0,color=red] {$[$};
		\node at (2,0.5) [rotate=0, color=red] {$1$};
		\draw[->] (0,1) -- (6,1) node [below]{$\mathbb{R}$};
		\draw  (2,1) -- (4,1)[ color=red] ;
		\node at (4,1) [rotate=0,color=red] {$]$};
		\node at (4,0.5) [rotate=0, color=red] {$3$};
	\end{tikzpicture}
\end{center}

\subsubsection*{Rechts halboffenes Intervall}
\[
[a,b[ := \{x\in\mathbb{R}\mid a\le x < b\}.
\]

\begin{center}
	\begin{tikzpicture}
		\node at (2,1) [rotate=0,color=red] {$[$};
		\node at (2,0.5) [rotate=0, color=red] {$a$};
		\draw[->] (0,1) -- (6,1) node [below]{$\mathbb{R}$};
		\draw  (2,1) -- (4,1)[ color=red] ;
		\node at (4,1) [rotate=0,color=black] {$[$};
		\node at (4,0.5) [rotate=0, color=black] {$b$};
	\end{tikzpicture}
	
	$b\notin[a,b[$
\end{center}

\noindent Analog: links halboffenes Intervall $]a,b]$.

\subsubsection*{Offenes Intervall}
\[
]a,b[ := \{x\in\mathbb{R}\mid a< x < b\}.
\]

\begin{center}
	\begin{tikzpicture}
		\node at (2,1) [rotate=0,color=black] {$]$};
		\node at (2,0.5) [rotate=0, color=black] {$a$};
		\draw[->] (0,1) -- (6,1) node [below]{$\mathbb{R}$};
		\draw  (2,1) -- (4,1)[ color=red] ;
		\node at (4,1) [rotate=0,color=black] {$[$};
		\node at (4,0.5) [rotate=0, color=black] {$b$};
	\end{tikzpicture}
\end{center}

\subsubsection*{Spezielle Teilintervalle}
Geometrisch entspricht dies einer Teilgeraden der reellen Zahlen:
\[
]-\infty,a[ := \{x\in\mathbb{R}\mid x<a\}.
\]

\begin{center}
	\begin{tikzpicture}
		\draw[<->] (0,1) -- (6,1) node [below]{$\mathbb{R}$};
		\draw  (0,1) -- (4,1)[ color=red] ;
		\node at (4,1) [rotate=0,color=black] {$[$};
		\node at (4,0.5) [rotate=0, color=black] {$a$};
	\end{tikzpicture}
\end{center}

\noindent Alternative Notation:
\[
]a,b[=(a,b),\quad [a,b[=[a,b),\quad ]a,b]=(a,b].
\]

\subsection{Durchschnitt, Vereinigung und Differenzmenge}

\subsubsection*{Durchschnitt}
\begin{center}
	\tikzset{
		schraffiert/.style={pattern=horizontal lines,pattern color=#1},
		schraffiert/.default=black
	}
	\begin{tikzpicture}
		\def\kreisUeins{(0.5,2) circle [radius=1.5]}
		\def\kreisUzwei{(1.5,2) circle [radius=1.5]}
		\begin{scope}
			\clip\kreisUeins;
			\draw[schraffiert=green]\kreisUzwei;
		\end{scope}
		\draw\kreisUeins;
		\draw\kreisUzwei;
		\node[anchor=east] at (-0.3,2) {$A$};
		\node[anchor=west] at (2.2,2) {$B$};
	\end{tikzpicture}
	
	$A\cap B$
\end{center}

\noindent Beispiel: Bilde $[1,2]\cap[2,3]$.
\[
[1,2]\cap[2,3]=\{2\}.
\]

\subsubsection*{Vereinigung}
\begin{center}
	\tikzset{
		schraffiert/.style={pattern=horizontal lines,pattern color=#1},
		schraffiert/.default=black
	}
	\begin{tikzpicture}
		\def\kreisUeins{(0.5,2) circle [radius=1.5]}
		\def\kreisUzwei{(1.5,2) circle [radius=1.5]}
		\draw[schraffiert=green]\kreisUzwei;
		\draw[schraffiert=green]\kreisUeins;
		\draw\kreisUeins;
		\draw\kreisUzwei;
		\node[anchor=east] at (-0.3,2) {$A$};
		\node[anchor=west] at (2.2,2) {$B$};
	\end{tikzpicture}
	
	$A\cup B$
\end{center}

\noindent Beispiel:
\[
[1,2]\cup[2,3]=[1,3].
\]

\subsubsection*{Differenzmenge}
\begin{center}
	\tikzset{
		schraffiert/.style={pattern=horizontal lines,pattern color=#1},
		schraffiert/.default=black
	}
	\begin{tikzpicture}
		\def\kreisUeins{(0.5,2) circle [radius=1.5]}
		\def\kreisUzwei{(1.5,2) circle [radius=1.5]}
		\begin{scope}
			\clip\kreisUeins;
			\draw[schraffiert=green]\kreisUzwei;
			\draw[schraffiert=green]\kreisUeins;
		\end{scope}
		\draw\kreisUeins;
		\draw\kreisUzwei;
		\fill[white] (1.5,2) circle (1.5);
		\draw (0.5,2) circle (1.5);
		\node[anchor=east] at (-0.3,2) {$A$};
		\node[anchor=west] at (2.2,2) {$B$};
	\end{tikzpicture}
	
	$A\setminus B$
\end{center}

\noindent Beispiel:
\[
[1,3]\setminus ]\tfrac{3}{2},\tfrac{5}{2}] = \left[1,\tfrac{3}{2}\right]\cup ]\tfrac{5}{2},3].
\]

\subsection{Aufgaben zu Mengen}

\subsubsection*{Aufgabe 1}
Bestimme die Menge
\[
\mathbb{M}=\{x\in \mathbb{Z}\mid x \text{ ist durch 3 teilbar}\}\cap \{x\in \mathbb{Z}\mid x \text{ ist durch 4 teilbar}\}.
\]
Zahlen, die durch 3 teilbar sind, lassen sich schreiben als $\{3k\mid k\in\mathbb{Z}\}$, und Zahlen, die durch 4 teilbar sind, als $\{4k\mid k\in\mathbb{Z}\}$.
Damit muss $x$ durch $\mathrm{lcm}(3,4)=12$ teilbar sein:
\[
\mathbb{M}=\{12k\mid k\in\mathbb{Z}\}.
\]

\subsubsection*{Aufgabe 2}
Bestimme die Menge
\[
\mathbb{M}=\bigcup_{m\in \mathbb{N}\setminus\{0\}} \{x\in \mathbb{Q}\mid m\cdot x\in \mathbb{Z}\}.
\]
Für festes $m$ gilt:
\[
\{x\in\mathbb{Q}\mid m x\in\mathbb{Z}\}=\left\{\frac{k}{m}\mid k\in\mathbb{Z}\right\}.
\]
Durch die Vereinigung über alle $m\ge 1$ entstehen genau alle rationalen Zahlen. Also:
\[
\mathbb{M}=\mathbb{Q}.
\]

\subsubsection*{Aufgabe 3}
Bestimme die Menge
\[
\mathbb{M}=\bigcap_{m\in \mathbb{R}}\{x\in\mathbb{R}\mid (x-m)(x-2)(x-3)=0\}.
\]
Für festes $m$ ist die Lösungsmenge $\{m,2,3\}$. In der Schnittmenge über alle $m\in\mathbb{R}$ bleiben nur die Elemente übrig, die \emph{für jedes} $m$ dabei sind:
\[
\mathbb{M}=\{2,3\}.
\]








